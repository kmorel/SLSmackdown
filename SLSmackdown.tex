% -*- latex -*-
\documentclass[letterpaper,twocolumn,10pt]{article}

\usepackage{color}
\definecolor{yellow}{rgb}{1,1,0}
\definecolor{black}{rgb}{0,0,0}
\definecolor{ltcyan}{rgb}{.75,1,1}
\definecolor{red}{rgb}{1,0,0}

\title{Sort-Last Smackdown!}
\author{Kenneth Moreland}

% Cite commands I use to abstract away the different ways to reference an
% entry in the bibliography (superscripts, numbers, dates, or author
% abbreviations).  \scite is a short cite that is used immediately after
% when the authors are mentioned.  \lcite is a full citation that is used
% anywhere.  Both should be used right next to the text being cited without
% any spacing.
\newcommand*{\lcite}[1]{~\cite{#1}}
\newcommand*{\scite}[1]{~\cite{#1}}

\newcommand*{\keyterm}[1]{\textbf{#1}}

\newcommand{\sticky}[1]{{\color{red}\textsc{[#1]}}}

\begin{document}
\maketitle

\begin{abstract}
  The only proven method for performing distributed-memory parallel
  rendering at large scales, tens of thousands of nodes, is a class of
  algorithms called sort last.  The fundamental operation of sort-last
  parallel rendering is an image composite, which combines a collection of
  images generated independently on each node into a single blended image.
  Over the years numerous image compositing algorithms have been proposed
  as well as several enhancements and rendering modes to these core
  algorithms.  However, the testing of these image compositing algorithms
  has been with an arbitrary set of enhancements, if any are applied at
  all.  In this paper we take a leading production-quality
  image-compositing framework, IceT, and use it as a testing framework for
  the leading image compositing algorithms of today.  The compositing
  enhancements provided by IceT, including some introduced in this paper,
  are employed in our measurements.  IceT also provides different
  compositing mechanisms for different rendering environments such as
  opaque surface versus volume rendering and fixed point versus floating
  point color representations.  These variations are also considered in our
  analysis.  To understand the behavior of these algorithms at vary large
  scale, we run tests on up to XXX cores of the Intrepid BlueGene/P at
  Argonne National Laboratories.
\end{abstract}

\section{Introduction}
\label{sec:Introduction}

The demands of parallel rendering continue to grow as visualization is
applied to ever larger scientific data.  Early efforts have demonstrated
the need on specialized visualization clusters containing hundreds of
nodes.  Because of recent constraints in building HEREHEREHERE

Although many aspects of parallel rendering have changed since the sorting
classification of parallel rendering algorithms was introduced \sticky{cite
  Molnar}, these classifications are still used today because they
accurately characterize and predict the scaling performance of these
algorithms.  When rendering on a hundred or more distributed nodes, the
most efficient class of algorithm is sort-last because it scales extremely
well with respect to the number of processes and size of the geometry being
rendered and because the main contributing factor to its overhead, the size
of the image being rendered, is fixed by the display that we are
using\sticky{cite 2001 CG\&A}.

\begin{itemize}
\item An introduction to a sort-last rendering framework named
  \keyterm{IceT} that is general perpose, production quality, and fully
  optimized.
\item The \keyterm{telescoping} algorithm, which can be applied to an
  existing image compositing algorithm that works best on powers of two,
  such as binary swap, to run efficiently on any number of processes.
\item A new method of \keyterm{image interlacing} that requires no
  additional image copying to reconstruct the final image.
\item A comparison of the ever popular binary-swap algorithm with the newer
  radix-k algorithm on leadership-class \sticky{correct?} high performance
  computers.  These tests are performed with every optimization one should
  expect in production quality parallel rendering as well as with a variety
  of rendering modes that can be encountered in production software.
\item An investigation comparing the performance of multi-tile compositing
  techniques \cite{2001 paper} with the best single image compositing
  techniques for single images.
\end{itemize}

\section{Previous Work}

binary swap, direct send, slic, 2-3 swap, radix-k.

run length encoding. image interlacing.  hybrid.  Wes' egpgv.

\end{document}

% -*- latex -*-

%%%%%%%%%%%%%%%%%%%%%%%%%%%%%%%%%%%%%%%%%%%%%%%%%%%%%%%%%%%%%%%%%%%%%%%%%%%%%
% This beginning part of the preamble is specific to the vgtc document class.

\documentclass{vgtc}                          % final (conference style)
%\documentclass[review]{vgtc}                 % review
%\documentclass[widereview]{vgtc}             % wide-spaced review
%\documentclass[preprint]{vgtc}               % preprint
%\documentclass[electronic]{vgtc}             % electronic version

%% Uncomment one of the lines above depending on where your paper is
%% in the conference process. ``review'' and ``widereview'' are for review
%% submission, ``preprint'' is for pre-publication, and the final version
%% doesn't use a specific qualifier. Further, ``electronic'' includes
%% hyperreferences for more convenient online viewing.

%% Please use one of the ``review'' options in combination with the
%% assigned online id (see below) ONLY if your paper uses a double blind
%% review process. Some conferences, like IEEE Vis and InfoVis, have NOT
%% in the past.

%% Figures should be in CMYK or Grey scale format, otherwise, colour 
%% shifting may occur during the printing process.

%% These three lines bring in essential packages: ``mathptmx'' for Type 1 
%% typefaces, ``graphicx'' for inclusion of EPS figures. and ``times''
%% for proper handling of the times font family.

\usepackage{mathptmx}
\usepackage{graphicx}
\usepackage{times}

%% We encourage the use of mathptmx for consistent usage of times font
%% throughout the proceedings. However, if you encounter conflicts
%% with other math-related packages, you may want to disable it.

%% If you are submitting a paper to a conference for review with a double
%% blind reviewing process, please replace the value ``0'' below with your
%% OnlineID. Otherwise, you may safely leave it at ``0''.
\onlineid{0}

%% declare the category of your paper, only shown in review mode
\vgtccategory{Research}

%% allow for this line if you want the electronic option to work properly
\vgtcinsertpkg

%% In preprint mode you may define your own headline.
%\preprinttext{To appear in an IEEE VGTC sponsored conference.}

%% Paper title.

\title{Sort-Last Smackdown!}

%% This is how authors are specified in the conference style

%% Author and Affiliation (single author).
%%\author{Roy G. Biv\thanks{e-mail: roy.g.biv@aol.com}}
%%\affiliation{\scriptsize Allied Widgets Research}

%% Author and Affiliation (multiple authors with single affiliations).
%%\author{Roy G. Biv\thanks{e-mail: roy.g.biv@aol.com} %
%%\and Ed Grimley\thanks{e-mail:ed.grimley@aol.com} %
%%\and Martha Stewart\thanks{e-mail:martha.stewart@marthastewart.com}}
%%\affiliation{\scriptsize Martha Stewart Enterprises \\ Microsoft Research}

%% Author and Affiliation (multiple authors with multiple affiliations)
%% \author{Roy G. Biv\thanks{e-mail: roy.g.biv@aol.com}\\ %
%%         \scriptsize Starbucks Research %
%% \and Ed Grimley\thanks{e-mail:ed.grimley@aol.com}\\ %
%%      \scriptsize Grimley Widgets, Inc. %
%% \and Martha Stewart\thanks{e-mail:martha.stewart@marthastewart.com}\\ %
%%      \parbox{1.4in}{\scriptsize \centering Martha Stewart Enterprises \\ Microsoft Research}}

\author{ %
  Kenneth Moreland\thanks{e-mail: kmorel@sandia.gov} %
    \scriptsize Sandia National Laboratories %
  \and Et All\thanks{e-mail: et.all@elsewhere.edu} %
    \scriptsize Elsewhere %
}

%% A teaser figure can be included as follows, but is not recommended since
%% the space is now taken up by a full width abstract.
%\teaser{
%  \includegraphics[width=1.5in]{sample.eps}
%  \caption{Lookit! Lookit!}
%}

%% Abstract section
\abstract{ The only proven method for performing distributed-memory
  parallel rendering at large scales, tens of thousands of nodes, is a
  class of algorithms called sort last.  The fundamental operation of
  sort-last parallel rendering is an image composite, which combines a
  collection of images generated independently on each node into a single
  blended image.  Over the years numerous image compositing algorithms have
  been proposed as well as several enhancements and rendering modes to
  these core algorithms.  However, the testing of these image compositing
  algorithms has been with an arbitrary set of enhancements, if any are
  applied at all.  In this paper we take a leading production-quality
  image-compositing framework, IceT, and use it as a testing framework for
  the leading image compositing algorithms of today.  The compositing
  enhancements provided by IceT, including some introduced in this paper,
  are employed in our measurements.  IceT also provides different
  compositing mechanisms for different rendering environments such as
  opaque surface versus volume rendering and fixed point versus floating
  point color representations.  These variations are also considered in our
  analysis.  To understand the behavior of these algorithms at vary large
  scale, we run tests on up to XXX cores of the Intrepid BlueGene/P at
  Argonne National Laboratories.  }

%% ACM Computing Classification System (CCS). 
%% See <http://www.acm.org/class/1998/> for details.
%% The ``\CCScat'' command takes four arguments.

\CCScatlist{
  \CCScat{I.3.1}{Computer Graphics}{Hardware Architecture}{Parallel processing}
}

%% Copyright space is enabled by default as required by guidelines.
%% It is disabled by the 'review' option or via the following command:
% \nocopyrightspace

% End of vgtc-specific portion of the preamble.
%%%%%%%%%%%%%%%%%%%%%%%%%%%%%%%%%%%%%%%%%%%%%%%%%%%%%%%%%%%%%%%%%%%%%%%%%%%%%

\usepackage{color}
\definecolor{yellow}{rgb}{1,1,0}
\definecolor{black}{rgb}{0,0,0}
\definecolor{ltcyan}{rgb}{.75,1,1}
\definecolor{red}{rgb}{1,0,0}

\author{Kenneth Moreland}

% Cite commands I use to abstract away the different ways to reference an
% entry in the bibliography (superscripts, numbers, dates, or author
% abbreviations).  \scite is a short cite that is used immediately after
% when the authors are mentioned.  \lcite is a full citation that is used
% anywhere.  Both should be used right next to the text being cited without
% any spacing.
\newcommand*{\lcite}[1]{~\cite{#1}}
\newcommand*{\scite}[1]{~\cite{#1}}

\newcommand*{\keyterm}[1]{\textbf{#1}}

\newcommand{\sticky}[1]{{\color{red}\textsc{[#1]}}}

\begin{document}

%% VGTC-specific:
%% The ``\maketitle'' command must be the first command after the
%% ``\begin{document}'' command. It prepares and prints the title block.

%% the only exception to this rule is the \firstsection command
\firstsection{Introduction}

\maketitle

%% \section{Introduction} 
\label{sec:Introduction}

The demands of parallel rendering continue to grow as visualization is
applied to ever larger scientific data.  Early efforts have demonstrated
the need on specialized visualization clusters containing hundreds of
nodes.  Because of recent constraints in building HEREHEREHERE

Although many aspects of parallel rendering have changed since the sorting
classification of parallel rendering algorithms was
introduced\lcite{Molnar1994}, these classifications are still used today
because they accurately characterize and predict the scaling performance of
these algorithms.  When rendering on a hundred or more distributed nodes,
the most efficient class of algorithm is sort-last because it scales
extremely well with respect to the number of processes and size of the
geometry being rendered and because the main contributing factor to its
overhead, the size of the image being rendered, is fixed by the display
that we are using\lcite{Wylie2001}.

\begin{itemize}
\item An introduction to a sort-last rendering framework named
  \keyterm{IceT} that is general perpose, production quality, and fully
  optimized.
\item The \keyterm{telescoping} algorithm, which can be applied to an
  existing image compositing algorithm that works best on powers of two,
  such as binary swap, to run efficiently on any number of processes.
\item A new method of \keyterm{image interlacing} that requires no
  additional image copying to reconstruct the final image.
\item A comparison of the ever popular binary-swap algorithm with the newer
  radix-k algorithm on leadership-class high-performance computers.  These
  tests are performed with every optimization one should expect in
  production quality parallel rendering as well as with a variety of
  rendering modes that can be encountered in production software.
\item An investigation comparing the performance of multi-tile compositing
  techniques\lcite{Moreland2001} with the best single image compositing
  techniques for single images.
\end{itemize}

\section{Previous Work}

binary swap, direct send, slic, 2-3 swap, radix-k.

run length encoding. image interlacing.  hybrid.  Wes' egpgv.

%% VGTC-specific section command.
\acknowledgements{Sandia National Laboratories is a multi-program
  laboratory operated by Sandia Corporation, a wholly owned subsidiary of
  Lockheed Martin Corporation, for the U.S. Department of Energy's
  National Nuclear Security Administration.}

\bibliographystyle{abbrv}
\bibliography{SLSmackdown}

\end{document}
